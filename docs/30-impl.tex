\chapter{Технологическая часть}

В данном разделе представлены средства разработки программного обеспечения, детали реализации и тестирование функций.

\section{Выбор языка программирования и среды разработки}
Для разработки программного продукта был выбран язык C++.  Данный выбор обусловлен следующими факторами.

\begin{enumerate}[label=\arabic*)]
	\item C++ обладает высокой вычислительной производительностью, что очень важно для выполнения поставленной задачи \cite{cplusplusperfomance}.
	\item C++ поддерживает парадигму объектно-ориентированного программирования. Можно представлять объекты сцены в виде объектов классов, а также пользоваться шаблонами проектирования \cite{isocplusplus}.
	\item Для С++ существует большое количество научной и учебной литературы по алгоритмам компьютерной графики ($\approx17000$ результатов запроса <<c++ computer graphics algorithms>>в поисковкой системе Академия Google).
\end{enumerate}

Для разработки программного продукта была выбрана среда разработки QT Creator. Данный выбор обусловлен следующими факторами.

\begin{enumerate}[label=\arabic*)]
	\item Основы работы с данной средой разработки изучались в рамках курса Программирования на Си.
	\item QT Creator позволяет работать с расширением Qt Designer, которое предоставляет инструменты для создания графического интерфейса \cite{qtdesigner}.
\end{enumerate}

\section{Формат входных и выходных данных и обоснование выбора}

Входными данными для разрабатываемого программного обеспечения является информация о сцене (о всех её объетов). Для представления входных данных был выбран текстовый файл формата OBJ, так как согласно \cite{3d} формат OBJ:
\begin{enumerate}[label=\arabic*)]
	\item не привязан к какой-либо программе, работающей с 3D моделированием;
	\item занимает третье место в рейтинге по количеству моделей (на 17.10.2022 первое место в рейтинге, указанным в статье);
	\item является читаемым и редактируемым форматом, в отличие от бинарных форматов, таких как 3DS и MAX, которые занимают первое и второе места соответственно в рейтинге по количеству моделей.
\end{enumerate}

Выходными данными является растровое изображение. Из всех возможных форматов выходных данных (BMP, GIF, JPG, JPEG, PNG, PBM, PGM, PPM, XBM, XPM), которая предоставляет библиотека Qt, был выбран формат PNG, так как:
\begin{enumerate}[label=\arabic*)]
	\item формат PNG является графическим, что было необходимым для создания фильма с помощью утилиты ffmpeg \cite{ffmpeg};
	\item формат PNG является платформонезависимым, в отличие от формата BMP;
	\item согласно \cite{jpgvspng}, изображения в формате PNG более качественные, чем изображения в формате JPG по метрике типа PSNR (peak signal-to-noise ratio, пиковое отношение сигнала к шуму).
\end{enumerate}

\section{Реализация алгоритмов}

В листинге \ref{lst:muscle} представлена структура объекта мышцы, а также реализация методов деформации и триангуляции. В листинге \ref{lst:cg} представлена реализация алгоритмов компьютерной графики: $z$-буфера и Гуро.


\section*{Вывод}

В данном разделе были рассмотрены средства, с помощью которых было реализовано ПО, а также представлены листинги кода с реализацией объекта мышцы и алгоритмов компьютерной графики.
