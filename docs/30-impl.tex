\chapter{Технологическая часть}

В данном разделе представлены средства разработки программного обеспечения, детали реализации и тестирование функций.

\section{Выбор языка программирования и среды разработки}
Для разработки программного продукта был выбран язык C++.  Данный выбор обусловлен следующими факторами.

\begin{enumerate}[label=\arabic*)]
	\item C++ обладает высокой вычислительной производительностью, что очень важно для выполнения поставленной задачи \cite{cplusplusperfomance}.
	\item C++ поддерживает парадигму объектно-ориентированного программирования. Можно представлять объекты сцены в виде объектов классов, а также пользоваться шаблонами проектирования \cite{isocplusplus}.
	\item Для С++ существует большое количество научной и учебной литературы по алгоритмам компьютерной графики ($\approx17000$ результатов запроса <<c++ computer graphics algorithms>>в поисковкой системе Академия Google).
\end{enumerate}

Для разработки программного продукта была выбрана среда разработки QT Creator. Данный выбор обусловлен следующими факторами.

\begin{itemize}
	\item Основы работы с данной средой разработки изучались в рамках курса Программирования на Си.
	\item QT Creator позволяет работать с расширением Qt Designer, которое предоставляет инструменты для создания графического интерфейса \cite{qtdesigner}.
\end{itemize}

\section{Реализация алгоритмов}

В листинге \ref{lst:muscle} представлена структура объекта мышцы, а также реализация методов деформации и триангуляции. В листинге \ref{lst:cg} представлена реализация алгоритмов компьютерной графики: $z$-буфера и Гуро.


\section*{Вывод}

В данном разделе были рассмотрены средства, с помощью которых было реализовано ПО, а также представлены листинги кода с реализацией объекта мышцы и алгоритмов компьютерной графики.
