\chapter*{Введение}
\addcontentsline{toc}{chapter}{Введение}

В современной жизни мы постоянно сталкиваемся с компьютерной графикой. Средства визуализации крайне важны для инженеров и архитекторов, огромную роль играет компьютерная графика в рекламе и индустрии развлечений. Без нее было бы невозможным создание многих компьютерных игр. Компьютерная графика используется в науке и промышленности для моделирования и визуализации различных физических процессов. Компьютерная графика развивается быстрыми темпами, постоянно появляются новые методы и алгоритмы, позволяющие показывать сложные и захватывающие эффекты, затрачивая для этого все меньше и меньше вычислительных ресурсов \cite{boreskov}.

 Например, алгоритмы компьютерной графики могут использоваться для визуализации решения задачи Стефана.

Целью данной работы является разработка программного обеспечения, позволяющего получить реалистичное изображение полого кусочка льда с жидкостью внутри.

Чтобы достигнуть поставленной цели, требуется решить следующие задачи:
\begin{enumerate}[label=\arabic*)]
    \item провести анализ алгоритмов построения реалистичных изображений;
    \item разработать метод построения реалистичного изображения полого кусочка льда с жидкостью внутри;
    \item реализовать метод построения реалистичного изображения полого кусочка льда с жидкостью внутри;
    \item исследовать зависимость времени выполнения однопоточной и многопоточной реализаций метода построения реалистичного изображения полого кусочка льда с жидкостью внутри от размера изображения.
\end{enumerate}
