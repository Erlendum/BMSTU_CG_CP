\chapter{Аналитическая часть}

В данном разделе проводится анализ задачи построения трёхмерного изображения сцены из объектов, рассматриваются различные методы, решающие данную задачу.

\section{Описание модели трёхмерного объекта на сцене}

Использование моделей позволяет точно отображать форму и размеры объекта на сцене.

Модели могут задаваться следующими способами:
\begin{itemize}
	\item Каркасная (проволочная) модель. В этой модели задается информация о вершинах и рёбрах объектов. Это простейший вид моделей. Этим видам модели присущ один, но весьма существенный недостаток: не всегда модель правильно передает представление об объекте.
	\item Поверхностная модель. Поверхность может описываться аналитически, либо может задаваться другим способом. Недостаток: отсутствует информация о том, с какой стороны поверхности находится материал.
	\item Объёмная модель. Эта форма модели отличается от поверхностной тем, что в объёмных моделях к информации
	о поверхности добавляется информация о том, с какой стороны расположен материал. Проще всего это можно сделать путём указания направления внутренней нормали.
\end{itemize}

Для решения поставленной задачи не подойдёт проволочная модель, так как в этом случае модель будет неправильно передавать представление об объекте. Поверхностная модель также не подойдёт, потому что нам необходимо знать, с какой стороны поверхности расположен материал. Таким образом, выбираем объёмную модель.

\section{Описание способа задания трёхмерного объекта на сцене}

Существует несколько способов задания объёмной модели:
\begin{itemize}
	\item Аналитический способ. Этот способ характеризуется описанием объекта в неявной форме, то есть для получения визуального представления нужно вычислять значение некоторой функции в различных точках пространства.
	\item Полигональная сетка. Это совокупность вершин, рёбер и граней, которые определяют форму многогранного объекта в трёхмерной компьютерной графике и объёмном моделировании. Гранями обычно являются треугольники, четырёхугольники или другие простые выпуклые многоугольники (полигоны). В данной работе грани имеют форму треугольников, так как любой полигон можно представить в виде треугольника.
	В свою очередь существуют различные способы хранения информации о полигональной сетке:
	\begin{itemize}
		\item Список граней. Характеризуется множеством граней и множеством вершин. В каждую грань входят как минимум три вершины.
		\item <<Крылатое>> представление. Представляет вершины, грани и ребра сетки. Это представление широко используется в программах для моделирования для предоставления высочайшей гибкости в динамическом изменении геометрии сетки, потому что могут быть быстро выполнены операции разрыва и объединения. Их основной недостаток -- высокие требования памяти и увеличенная сложность из-за содержания множества индексов.
		\item Вершинное представление. Описывает объект как множество вершин, соединенных с другими вершинами. Это простейшее представление, но оно не широко используемое, так как информация о гранях и ребрах не выражена явно. Поэтому нужно обойти все данные чтобы сгенерировать список граней для рендеринга. Кроме того, не легко выполняются операции на ребрах и гранях.
	\end{itemize}
\end{itemize}

При выборе способа задания объекта в курсовой работе определяющим фактором стала скорость выполнения геометрических преобразований.

Оптимальное представление -- полигональная сетка. Такая модель позволит легко описывать сложные объекты сцены. 

Способом хранения информации о полигональной сетке был выбран список граней, потому что этот способ даёт явное описание граней, что позволяет эффективно выполнять геометрические преобразования над объектами сцены.

\section{Формализация объектов синтезируемой сцены}

Синтезируемая сцена состоит из следующих объектов:
\begin{enumerate}[label=\arabic*)]
	\item Геометрический объект -- представляется в виде полигональной сетки. Для описания геометрического объекта требуется указать координаты вершин, связи вершин (рёбра), фоновое освещение (цвет), диффузное освещение (цвет), зеркальное освещение (цвет), коэффициент фонового освещения, коэффициент диффузного освещения, коэффициент зеркального освещения, степень, аппроксимирующая пространственное распределение зеркально отражённого света, коэффициент отражения, коэффициент преломления, показатель преломления.
	\item Источник света -- представляется в виде вектора. Также к характеристикам источника света относятся расположение, цвет и интенсивность излучения.
	\item Камера -- характеризуется пространственным положением и направлением взгляда.
\end{enumerate}

\section{Анализ задачи построения трёхмерного изображения сцены из объектов}

Построение трёхмерного изображения сцены заключается в преобразовании объектов этой сцены
в изображение на растровом дисплее. Для создания реалистичного изображения необходимо учитывать несколько факторов:
\begin{itemize}
	\item Удаление невидимых линий и поверхностей.
	\item Учёт теней.
	\item Учёт освещения. 
	\item Учёт свойств материалов объекта: способность отражать свет, способность преломлять свет.
\end{itemize}

\subsection{Удаление невидимых линий и поверхностей}

Задача удаления невидимых линией и поверхностей является одной из наиболее сложных в компьютерной графике. 
Алгоритмы удаления невидимых линий и поверхностей служат для определения линий ребер, поверхностей или объемов, которые видимы или невидимы для наблюдателя, находящегося в заданной точке пространства.

Алгоритмы удаления невидимых линий и поверхностей делятся на:
\begin{itemize}
	\item Алгоритмы, работающие в объектном пространстве (мировая система координат, высокая точность).
	\item Алгоритмы, работающие в пространстве изображений (система координат связана с дисплеем, точность ограничена разрешающей способностью дисплея).
\end{itemize}
Рассмотрим алгоритмы удаления невидимых линий и поверхностей.

\subsubsection{Алгоритм Робертса}

Данный алгоритм работает в объектном пространстве, решая задачу только с выпуклыми телами.

Алгоритм выполняется в три этапа.

\begin{enumerate}[label=\arabic*)]
	\item Этап подготовки исходных данных.
	На данном этапе должна быть задана информация о телах. Для каждого тела сцены должна быть сформирована матрица тела $V$. Размерность матрицы -- $4 * n$, где $n$ -- количество граней тела.
	
	Каждый столбец матрицы представляет собой четыре коэффициента уравнения плоскости $ax + by + cz + d = 0$, проходящей через очередную грань.
	
	Таким образом, матрица тела будет представлена в следующем виде:
	
	\begin{equation}
		V = \begin{pmatrix}
			a_{1} & a_{2} & \ldots & a_{n}\\
			b_{1} & b_{2} & \ldots & b_{n}\\
			c_{1} & c_{2} & \ldots & c_{n}\\
			d_{1} & d_{2} & \ldots & d_{n}
		\end{pmatrix}
	\end{equation}
	
	Матрица тела должна быть сформирована корректно, то есть любая точка, расположенная внутри тела, должна располагаться по положительную сторону от каждой грани тела. В случае, если для очередной грани условие не выполняется, соответствующий столбец матрицы надо умножить на $-1$. Для проведения проверки следует взять точку, расположенную внутри тела. Координаты такой точки можно получить путем усреднения координат всех вершин тела.
	\item Этап удаления ребер, экранируемых самим телом.
	На данном этапе рассматривается вектор взгляда $E = \{0, 0, -1, 0\}$.
	Для определения невидимых граней достаточно умножить вектор $E$ на матрицу тела $V$. Отрицательные компоненты полученного вектора будут соответствовать невидимым граням.

	\item Этап удаления невидимых ребер, экранируемых другими телами сцены.
	На данном этапе для определения невидимых точек ребра требуется построить луч, соединяющий точку наблюдения с точкой на ребре. Точка будет невидимой, если луч на своем пути встречает в качестве преграды рассматриваемое тело. Если тело является преградой, то луч должен пройти через тело. Если луч проходит через тело, то он находится по положительную сторону от каждой грани тела.
\end{enumerate}

Преимущества алгоритма Робертса:
\begin{itemize}
	\item алгоритм работает в объектном пространстве, точность вычислений высокая.
\end{itemize}

Недостатки алгоритма Робертса:
\begin{itemize}
	\item теоретический рост сложности алгоритма -- квадрат числа объектов. Для решения данной проблемы достаточно воспользоваться модифицированными реализациями, например, с использованием габаритных тестов или сортировки по оси z;
	\item все тела сцены должны быть выпуклыми. Данная проблема также приводит к усложнению алгоритма, так как потребуется прибегнуть к проверке объектов на выпуклость и их разбиению на выпуклые многоугольники.
\end{itemize}

Таким образом, алгоритм Робертса не подходит для решения поставленной задачи по следующим причинам.

\begin{itemize}
	\item Возникновение проблем при наличии множества невыпуклых тел на сцене.
	\item Невозможность визуализации зеркальных поверхностей.
\end{itemize}

\subsubsection{Алгоритм, использующий z-буфер}

Данный алгоритм работает в пространстве изображения. Используется два буфера:
\begin{itemize}
	\item буфер кадра, в котором хранятся атрибуты каждого пикселя в пространстве изображения;
	\item z-буфер, куда помещается информация о координате $z$ для каждого пикселя.
\end{itemize}

Первоначально в z-буфере находятся минимально возможные значения z, а в буфере кадра располагаются пиксели, описывающие фон. Каждый многоугольник преобразуется в растровую форму и записывается в буфер кадра.

В процессе подсчета глубины нового пикселя, он сравнивается с тем значением, которое уже лежит в z-буфере. Если новый пиксель расположен ближе к наблюдателю, чем предыдущий, то он заносится в буфер кадра и происходит корректировка z-буфера.

Для решения задачи вычисления глубины $z$ каждый многоугольник описывается уравнением $ax + by + cz + d = 0$. При $c = 0$ многоугольник для наблюдателя вырождается в линию. 

Для некоторой сканирующей строки $y = const$, поэтому имеется возможность рекуррентно высчитывать $z^\prime$ для каждого $x^\prime = x + dx$:

\begin{equation}
	z^\prime - z = -\frac{ax^\prime + d}{c} +\frac{ax + d}{c} = \frac{a(x - x^\prime)}{c}
\end{equation}

Получим: $z^\prime = z - \frac{a}{c^\prime}$, так как $x - x^\prime = dx = 1$.

При этом стоит отметить, что для невыпуклых многогранников предварительно потребуется удалить нелицевые грани.

Преимущества алгоритма, использующего z-буфер:
\begin{itemize}
	\item простота реализации;
	\item алгоритм имеет линейную сложность.
\end{itemize}

Недостатки алгоритма, использующего z-буфер:
\begin{itemize}
	\item большой объем требуемой памяти;
	\item реализация эффектов прозрачности сложна;
	\item дополнительные вычислительные операции в случае невыпуклых тел.
\end{itemize}

Таким образом, алгоритм, использующий z-буфер не подходит для решения поставленной задачи по следующим причинам.

\begin{itemize}
	\item Сложность визуализации прозрачных поверхностей.
	\item Невозможность визуализации зеркальных поверхностей.
\end{itemize}

\subsubsection{Алгоритм обратной трассировки лучей}

Метод прямой и обратной трассировки \cite{ray} заключается в том, что от момента испускания лучей
источником света до момента попадания в камеру, траектории лучей отслеживаются, и рассчитываются
взаимодействия лучей с лежащими на траекториях объектами. Луч может быть поглощен, диффузно или
зеркально отражен или, в случае прозрачности некоторых объектов, преломлен.
Ray tracing – метод расчета глобального освещения, рассматривающий освещение, затенение
(расчет тени), многократные отражения и преломления.
Для расчета теней применяют методы прямой и обратной трассировки лучей.
Метод прямой трассировки предполагает построение траекторий лучей от всех источников
освещения ко всем точкам всех объектов сцены. Это, так называемые, первичные лучи. Точки,
лежащие на противоположной от источника света стороне, исключаются из расчета. Для всех
остальных точек вычисляется освещенность с помощью локальной модели освещения. Если
объект не является отражающим или прозрачным, то траектория луча на этой точке обрывается. Если
же поверхность объекта обладает свойством отражения (reflection) или преломления (refraction), то из
точки строятся новые лучи, направления которых совершенно точно определяются законами
отражения и преломления. Траектории новых лучей также отслеживаются. Построение новых
траекторий и расчеты ведутся до тех пор, пока все лучи либо попадут в камеру, либо выйдут за
пределы видимой области. Очевидно, что при прямой трассировке лучей мы вынуждены
выполнять расчеты для лучей, которые не попадут в камеру, то есть проделывать бесполезную
работу. По некоторым оценочным данным, доля таких «слепых» лучей довольно велика. Эта
главная, хотя и далеко не единственная причина того, что метод прямой трассировки лучей
считается неэффективным и на практике не используется.
Алгоритм обратной трассировки лучей - основной способ расчета освещенности методом
трассировки лучей. Метод трассировки лучей – первый метод расчета глобальной освещенности,
учитывающий взаимное влияние объектов сцены друг на друга.

Алгоритм трассировки лучей требует большого количества вычислений, поскольку он
предполагает поиск пересечений всех объектов сцены со всеми лучами, количество которых
равно размеру растра. Поэтому время синтеза изображения оказывается очень большим.
Однако из описания алгоритма видно, что вычисление цвета каждого пиксела может быть
выполнено параллельно. Такая оптимизация позволит значительно ускорить синтез изображения.

Преимущества алгоритма обратной трассировки лучей:
\begin{itemize}
	\item высокая реалистичность синтезируемого изображения;
	\item реализация алгоритма предполагает учёт теней;
	\item простота визуализации зеркальных и прозрачных поверхностей.
\end{itemize}

Недостатки алгоритма обратной трассировки лучей:
\begin{itemize}
	\item производительность.
\end{itemize}

\subsubsection{Вывод}

Таким образом, в качестве алгоритма удаления невидимых рёбер и поверхностей был выбран алгоритм обратной трассировки лучей из-за высокой реалистичного синтезируемого изображения и возможности визуализации зеркальных и прозрачных поверхностей.

\subsection{Учёт теней}

При использовании алгоритма обратной трассировки лучей, выбранного в качестве алгоритма удаления невидимых рёбер и поверхностей, построение теней происходит по ходу выполнения алгоритма: пиксель будет затемнён, если луч испускаемый из точки попадания первичного луча, испускаемого из камеры, попадает на другой объект.

\subsection{Учёт освещения}

Модель освещения предназначена для того, чтобы рассчитать интенсивность отражённого к наблюдателю света в каждой точке (пикселе) изображения. Глобальная модель учитывает не только свет и ориентацию поверхностей, но также и свет, отражённый от других объектов (или пропущенный через них) Благодаря этому глобальная модель освещённости способна воспроизводить эффекты зеркального отражения и преломления лучей (прозрачность и полупрозрачность), а также затенение, что является необходимым для решения поставленной в курсовой работе задачи. Глобальная модель является составной частью алгоритма удаления невидимых рёбер и поверхностей с помощью обратной трассировки лучей.

Глобальная модель освещения для каждого пикселя изображения определяет его интенсивность. Сначала определяется непосредственная освещённость источниками без учёта отражений от других поверхностей (вторичная освещённость): отслеживаются лучи, направленные ко всем источникам. Тогда наблюдаемая интенсивность (или отражённая точкой энергия) выражается следующим соотношением:

\begin{equation}
	I = k_0I_0 + k_d\sum\limits_{j}^{}I_j(n \cdot l_j) + k_r\sum\limits_{j}^{}I_j(s \cdot r_j)^{\beta}+k_rI_r+k_tI_t,
\end{equation}
где

$k_0$ -- коэффициент фонового (рассеянного) освещения,

$k_d$ -- коэффициент диффузного отражения,

$k_r$ -- коэффициент зеркального отражения,

$k_t$ -- коэффициент пропускания,

$n$ -- единичный вектор нормали к поверхности в точке,

$l_j$ -- единичный вектор, направленный к $j$-му источнику света,

$s$ -- единичный локальный вектор, направленный в точку наблюдения,

$r_j$ -- отражённый вектор $l_j$,

$I_0$ -- интенсивность фонового освещения,

$I_j$ -- интенсивность $j$-го источника света,

$I_r$ --  интенсивность, приходящая по зеркально отражённому лучу,

$I_t$ -- интенсивность, приходящая по преломлённому лучу.

В алгоритме удаления невидимых рёбер и поверхностей трассировка луча продолжалась до первого пересечения с поверхностью. В глобальной модели освещения этим дело не ограничивается: осуществляется дальнейшая трассировка отражённого и преломлённого лучей. Таким образом, происходит разветвление алгоритма в виде двоичного дерева. Процесс продолжается до тех пор, пока очередные лучи не останутся без пересечений. Отражение и преломление рассчитываются по законам геометрической оптики. Теоретически дерево может оказаться бесконечным, поэтому при его построении желательно задать максимальную глубину, чтобы избежать переполнения памяти компьютера.

\section{Вывод}
Были рассмотрены способы задания трёхмерных моделей и выбрана объёмная форма задания моделей.

Также были рассмотрены алгоритмы удаления невидимых рёбер и поверхностей:
\begin{itemize}
	\item алгоритм Робертса;
	\item алгоритм, использующий z-буфер;
	\item алгоритм обратной трассировки лучей.
\end{itemize}
В качестве реализуемого был выбран алгоритм обратной трассировки лучей с глобальной моделью освещения.
 
