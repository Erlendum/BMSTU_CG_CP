\chapter{Конструкторская часть}

В данном разделе представлены математические основы алгоритма обратной трассировки лучей, разработка алгоритма обратной трассировки лучей, разработка и обоснование используемых типов и структур данных и разработка структуры программного комплекса.

\section{Математические основы алгоритма обратной трассировки лучей}

\subsection{Поиск пересечения луча с полигонами}

Для поиска пересечения луча с полигонами используется барицентрический тест. Это самый известный тест на пересечение <<луч-треугольник>>. Имея три точки на плоскости, можно выразить любую другую точку через ее барицентрические координаты.

Пусть луч $R(t)$ с началом в точке $O$ и нормализованным вектором направления $D$ определяется как:
\begin{equation}
	R(t) = O + tD
\end{equation}

Пусть вершины полигона обозначаются как $V_0, V_1, V_2$
Тогда, точка $T(u,v)$ в полигоне задаётся выражением:
\begin{equation}
	T(u,v) = (1 - u - v)V_0 + uV_1 + vV_2,
\end{equation}
где $(u,v)$ -- барицентрические координаты ($u \geq 0, v \geq 0, u + v \leq 1$)

Вычисление пересечения луча $R(t)$ и треугольника эквивалентно решению уравнения
$R(t) = T(u, v)$. В этом случае получим:
\begin{equation}
	O + tD = (1- u - v)V_0 + uV_1 + vV_2
\end{equation}

В матричном виде:
\begin{equation}
\label{slau}
\begin{bmatrix}
	-D & V_1 - V_0, V_2 - V_0
\end{bmatrix}
\begin{bmatrix}
t\\
u\\
v
\end{bmatrix} = O - V_0 
\end{equation}

Это означает, что барицентрические координаты ($u,v$) и расстояние $t$ от точки испускания луча до точки пересечения луча с полигоном могут быть найдены путём решения СЛАУ, которая написана выше.

Вышесказанное можно рассматривать геометрически как перевод полигона (треугольника) в начало координат и преобразование его в в треугольник с единичными длинами по $y$ и $z$ с направлением луча по оси $x$. Это показано на рисунке
\ref{img:intersection}.

\img{70mm}{intersection}{Геометрическая интерпретация СЛАУ}


Пусть $E_1=V_1-V_0, E_2=V_2-V_0$ и $T=O - V_0$. Тогда решим \ref{slau}, используя метод Крамера:

\begin{equation}
\label{solution}
\begin{bmatrix}
t\\
u\\
v
\end{bmatrix} = \frac{1}{(D\times E_2) \cdot E_1}
\begin{bmatrix}
(T\times E_1) \cdot E_2\\
(D\times E_2) \cdot T\\
(T\times E_1) \cdot D
\end{bmatrix} = \frac{1}{P \cdot E_1}
\begin{bmatrix}
Q \cdot E_2\\
P \cdot T\\
Q \cdot D
\end{bmatrix},
\end{equation}
где $P = D \times E_2$ и $Q = T \times E_1$.

\subsection{Поиск нормали к полигонам}

Для поиска нормали к полигонам необходимо найти векторное произведение двух векторов, которые лежат на полигоне:
\begin{equation}
N = (V_2 - V_0) \times (V_1 - V_0),
\end{equation}
где $V_0, V_1, V_2$ -- вершины полигона.

\subsection{Поиск направления преломлённого и отражённого лучей}

Для алгоритма обратной трассировки лучей нужно уметь находить отражённый и преломлённый лучи, при этом учитывая модель освещения Уиттеда.
Отражённый луч можно найти, зная направление падающего луча и нормаль к поверхности. 

Пусть $L$ -- направление луча, а $n$ -- нормаль к поверхности. 
Луч можно разбить на две части: $L_p$ которая перпендикулярна нормали, и  $L_n$ – параллельна нормали.

Представленная ситуация изображена на рисунке 2.4:


Учитывая свойства скалярного произведения $L_n = n \cdot (n, L)$ и  $L_p = L - n \cdot (n,L)$
Так как отражённый луч выражается через разность этих векторов, то отражённый луч выражается по формуле \ref{reflect_ray}:
\begin{equation}
\label{reflect_ray}
R = 2 \cdot n \cdot (n, L) - L
\end{equation}

По закону преломления падающий, преломлённый луч и нормаль к поверхности лежат в одной плоскости. 
Пусть $ \mu_i$ -- показатели преломления сред, а $\eta_i$ – углы падения и отражения света соответственно. 
Применяя закон Снеллиуса, параметры преломлённого луча можно вычислить по формуле \ref{refract_ray}:
\begin{equation}
\label{refract_ray}
\begin{aligned}
R = \frac{\mu_1}{\mu_2} L + ( \frac{\mu_1}{\mu_2} cos(\eta_1) - cos(\eta_2))n ,
\end{aligned}
\end{equation} 
где $cos(\eta_2) = \sqrt{1 - (\frac{\mu_1}{\mu_2})^2 \cdot (1 - cos(\eta_1))^2}$
\section{Разработка алгоритмов}

Для алгоритмов, разработанных автором работы, представлены схемы алгоритмов. Для алгоритма синтеза изображения, использующего в своей основе алгоритмы Z-буфера\cite{zbuf} и Гуро\cite{lmodels} представлена блок-схема.

\subsection{Алгоритм деформации мышцы}

На рисунке \ref{img:deform} представлена схема алгоритма деформации мышцы.

\img{150mm}{deform}{Схема алгоритма деформации мышцы}

\clearpage
\subsection{Алгоритм триангуляции мышцы}


На рисунке \ref{img:triangulation} представлена схема алгоритма триангуляции мышцы.

\img{190mm}{triangulation}{Схема алгоритма триангуляции мышцы}

\clearpage
\subsection{Алгоритм синтеза изображения}

На рисунке \ref{img:graphics} представлена блок-схема алгоритма синтеза изображения.

\img{190mm}{graphics}{Блок-схема алгоритма синтеза изображения}

\section*{Вывод}

В данном разделе были представлены требования к программному обеспечению и разработаны схемы реализуемых алгоритмов.
