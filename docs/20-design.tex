\chapter{Конструкторская часть}

В данном разделе представлены математические основы алгоритма обратной трассировки лучей, разработка алгоритма обратной трассировки лучей, разработка и обоснование используемых типов и структур данных и разработка структуры программного комплекса.

\section{Математические основы алгоритма обратной трассировки лучей}

\subsection{Поиск пересечения луча с полигонами}

Для поиска пересечения луча с полигонами используется барицентрический тест. Это самый известный тест на пересечение <<луч-треугольник>>. Имея три точки на плоскости, можно выразить любую другую точку через ее барицентрические координаты.

Пусть луч $R(t)$ с началом в точке $O$ и нормализованным вектором направления $D$ определяется как:
\begin{equation}
	R(t) = O + tD
\end{equation}

Пусть вершины полигона обозначаются как $V_0, V_1, V_2$
Тогда, точка $T(u,v)$ в полигоне задаётся выражением:
\begin{equation}
	T(u,v) = (1 - u - v)V_0 + uV_1 + vV_2,
\end{equation}
где $(u,v)$ -- барицентрические координаты ($u \geq 0, v \geq 0, u + v \leq 1$)

Вычисление пересечения луча $R(t)$ и треугольника эквивалентно решению уравнения
$R(t) = T(u, v)$. В этом случае получим:
\begin{equation}
	O + tD = (1- u - v)V_0 + uV_1 + vV_2
\end{equation}

В матричном виде:
\begin{equation}
\begin{bmatrix}
	-D & V_1 - V_0, V_2 - V_0
\end{bmatrix}
\begin{bmatrix}
t\\
u\\
v
\end{bmatrix} = O - V_0 
\end{equation}

Это означает, что барицентрические координаты ($u,v$) и расстояние $t$ от точки испускания луча до точки пересечения луча с полигоном могут быть найдены путём решения СЛАУ, которая написана выше.


\section{Разработка алгоритмов}

Для алгоритмов, разработанных автором работы, представлены схемы алгоритмов. Для алгоритма синтеза изображения, использующего в своей основе алгоритмы Z-буфера\cite{zbuf} и Гуро\cite{lmodels} представлена блок-схема.

\subsection{Алгоритм деформации мышцы}

На рисунке \ref{img:deform} представлена схема алгоритма деформации мышцы.

\img{150mm}{deform}{Схема алгоритма деформации мышцы}

\clearpage
\subsection{Алгоритм триангуляции мышцы}


На рисунке \ref{img:triangulation} представлена схема алгоритма триангуляции мышцы.

\img{190mm}{triangulation}{Схема алгоритма триангуляции мышцы}

\clearpage
\subsection{Алгоритм синтеза изображения}

На рисунке \ref{img:graphics} представлена блок-схема алгоритма синтеза изображения.

\img{190mm}{graphics}{Блок-схема алгоритма синтеза изображения}

\section*{Вывод}

В данном разделе были представлены требования к программному обеспечению и разработаны схемы реализуемых алгоритмов.
